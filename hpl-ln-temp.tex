%HU, Pili
%Create: 20110910
%Modify: 20110910
%use this template to facilitate speedup noting


%these initial settings are copied from 
%http://www-solar.mcs.st-and.ac.uk/~clare/Latex/latex_template.tex
\documentclass[11pt]{article}
\topmargin-0.1cm 
\headsep0.1cm 
\headheight0.1cm 
\footskip1.0cm
\oddsidemargin0cm 
\evensidemargin0cm 
\textwidth15cm
\textheight23cm

\usepackage{graphicx}
\usepackage[square]{natbib}
\usepackage{amsmath}
\usepackage{framed}

\newenvironment{hint}
{\begin{center}\begin{minipage}{0.55\linewidth}\begin{framed}\it}
{\end{framed}\end{minipage}\end{center}}

%HU, Pili
%Create: 20110910
%Modify: 20110910
%purpose of this file is to gather commonly used
%mathematical abbreviations, to speed up writing
%notes

%the following commands are not originated by me
%I pick them from http://www-solar.mcs.st-and.ac.uk/~clare/Latex/
\newcommand{\diff}[2]{\frac{{\rm d}#1}{{\rm d}#2}}
\newcommand{\ndiff}[3]{\frac{{\rm d}^{#3}#1}{{\rm d}#2^{#3}}}
\newcommand{\pdiff}[2]{\frac{\partial #1}{\partial #2}}
\newcommand{\npdiff}[3]{\frac{\partial^{#3} #1}{\partial #2^{#3}}}
\newcommand{\e}[1]{\ensuremath{{\rm e}^{#1}}}
\newcommand{\ldiff}[2]{\ensuremath{{\rm d}#1/{\rm d}#2}}
\newcommand{\lpdiff}[2]{\ensuremath{\partial#1/\partial#2}}
\newcommand{\lnpdiff}[3]{\ensuremath{\partial^{#3}#1/\partial#2^{#3}}}




\renewcommand{\maketitle}{
	aaa
%		\@title
%		\@author
}

%===============begin document=======

\begin{document}
\title{\sc A Template Package for Lecture Notes}
\author{Pili HU}

%\maketitle

\begin{abstract}
This document\citep{SampleBib} aims at providing a quick template for lecture notes. Along with this tex file, you'll find a ".bib", "Makefile", "commands.tex" in the same directory.  

\end{abstract}

\section{Introduction}

%\begin{hint}
%\begin{center}
%\begin{minipage}{8cm}
%\end{minipage}
%\end{center}
%\end{hint}

\begin{hint}
This small portion of article shows the hint label. You can create as many hints as you like using this template. 
\end{hint}

\subsection{Layout}
Along with this tex file, you'll find these files:
\begin{itemize}
	\item \verb!hpl-ln-temp.bib! : the bibliagraphy database
	\item \verb!Makefile! : if you have linux programing experience, you must be familiar with this file. \textnormal{\^\_\^}
	\item \verb!commands.tex! : this file provides a new set of commands that will be helpful to take note
\end{itemize}

\subsection{Compile}
Much simpler than any tutorial! Just type:
\begin{verbatim}
make
\end{verbatim}
Feel it disgusting to delete temporary files manually? Try this:
\begin{verbatim}
make clean
\end{verbatim}
Your world is clean.

\section{Conclusion}

This is conclusion.

%\citet{SampleBib} \\
%\cite{SampleBib}  \\

\bibliographystyle{plainnat}    
%NOTICE: Specify your bibliagraphy file here
\bibliography{hpl-ln-temp}       

\end{document}
